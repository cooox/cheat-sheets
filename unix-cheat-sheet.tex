\documentclass[landscape, a4paper]{article}

\usepackage[landscape, margin=1cm]{geometry}
\usepackage[utf8]{inputenc}
\usepackage[T1]{fontenc}
\usepackage{fontspec}
\usepackage[english]{babel}
\usepackage{multicol}
\usepackage[os=win]{menukeys}
\usepackage{textcomp}
\usepackage{changepage}
\usepackage{titlesec}
\usepackage{enumitem}

\setmainfont{Cantarell}
\setmonofont{DejaVu Sans Mono}

\pagenumbering{gobble}
\setlength{\parindent}{0pt}
\titlespacing\section{0pt}{3pt}{0pt}
\titlespacing\subsection{0pt}{3pt}{0pt}
\setlist{nosep}

\newcommand{\cl}[1]{\texttt{#1}}
\newcommand{\cv}[1]{\textit{\texttt{#1}}}
\newcommand{\shcmd}[2]{\texttt{\textbf{#1} #2}}

\begin{document}
\begin{center}
    \cl{The ultimate programming course cheat sheet for UNIX-like operating system newbies}
\end{center}
\begin{multicols*}{3}
\section*{\centering Directories and files}
\subsection*{Concepts}
\begin{itemize}
    \item Files are organised in a directory tree/hierarchy
    \item Everything is a file (e.g. keyboard, printers, ...)
    \item Each process has access to the files \textit{stdin} (input),\\
          \textit{stdout} (buffered output), \textit{stderr} (unbuffered output)
    \item Each process operates in a \textit{working directory}
    \item Each user has a \textit{home directory}
\end{itemize}
\subsection*{Paths}
\textit{Path} = Identifier for the location of file/directory
\begin{itemize}
    \item Paths consists of a parent directory list + file/directory
    \item Files and directories are separated by a '\cl{/}
    \item Directory paths may contain a trailing '\cl{/}'
\end{itemize}
\textit{Absolute path} = Full location (first character = '\cl{/}')\\
\textit{Relative path} = Relative location (first character $\neq$ '\cl{/}')
\begin{tabular}{ll}
\cl{.}              & path to the directory itself\\
\cl{..}             & path to the parent directory\\
\cl{/usr/bin/ls}    & example for an absolute file path\\
\cl{/home/foo/}     & example for an absolute directory path\\
\cl{./a.out}        & example for a relative file path\\
\end{tabular}
\subsection*{File system hierarchy}
\begin{tabular}{ll}
\cl{/}          & Root directory\\
\cl{/bin}       & Essential command executables\\
\cl{/dev}       & Device files\\
\cl{/etc}       & System-wide configuration files\\
\cl{/opt}       & Manually added software\\
\cl{/sbin}      & Essential administrative executables\\
\cl{/tmp}       & Temporary files\\
\cl{/usr}       & System resources for users\\
\cl{/usr/bin}   & Command executables\\
\cl{/usr/local} & Site-local data\\
\cl{/usr/sbin}  & Administrative executables\\
\cl{/var}       & Variable files\\
\end{tabular}
\shcmd{man}{hier} (or \shcmd{man}{file-hierarchy} on recent Linux distributions) to get a more detailed overview
\section*{\centering Terminal (emulator)}
\textit{Text terminal} = Computer interface for text entry/display\\
\textit{Terminal emulator} = Application that emulates a \textit{text terminal} in a graphical environment\\
Examples for terminal emulators: \cl{xterm}, \cl{urxvt}, \cl{guake}

\subsection*{Opening a terminal}
\begin{tabular}{ll}
Unity/GNOME     & \keys{Ctrl + Alt + T}\\
Mac OS          & \keys{Cmd + \Space} $\rightarrow$ ``terminal'' $\rightarrow$ \keys{\return}\\
Bash on Windows & \keys{Win + R} $\rightarrow$ ``bash'' $\rightarrow$ \keys{\return}
\end{tabular}
\section*{\centering Shell}
\textit{Unix shell} = User interface that accepts commands to operate a computer\\
\shcmd{man}{intro} to get an introduction into basic shell usage\\

Examples for shell programs: \cl{sh}, \cl{bash}, \cl{zsh}, \cl{fish}, \cl{ksh}
\subsection*{Prompt}
\textit{Prompt} = Text sequence that precedes each line that prompts the user to enter a command\\
\begin{tabular}{ll}
\cl{[foo@bar /var/www]\$ } & example prompt in bash\\
\end{tabular}

$\Rightarrow$ \textit{user} \cl{foo} is operating in the \textit{working directory} \cl{/var/www} at the computer with the \textit{host name} \cl{bar}
\subsection*{Line editing}
\begin{tabular}{ll}
\keys{Ctrl + A}/\keys{E} & Go to beginning/end of the line\\
\keys{Ctrl + U}/\keys{K} & Erase to beginning/end of the line\\
\keys{Alt + B}/\keys{F}  & Go one word forward/backward in line\\
\keys{Ctrl + W}          & Erase to beginning of current word\\
\keys{Ctrl + C} & Cancel the current command line
\end{tabular}
\subsection*{Metacharacters}
The following characters have special meaning and sometimes they can't be used directly as arguments/words:\\
\cl{| \& ; < > ( ) \$ \textasciigrave{} \textbackslash " ' * ? [ ] \# \textasciitilde{} = \% ! \{ \}
\keys{\Space} \keys{\tab} \keys{\return}}\\
Their special meaning can be disabled:\\
\begin{tabular}{ll}
\cl{\textbackslash} & preserves the literal value of the following character\\
\cl{' '}            & preserves the literal values of enquoted characters\\
\cl{" "}            & like \cl{' '} but characters \cl{\textasciigrave{}  \$ \textbackslash} retain their meaning
\end{tabular}
\subsection*{Expansions}
\begin{tabular}{ll}
\cl{\textasciitilde}    & \textit{home directory} of the current user\\
\cl{*}                  & matches any character sequence\\
\cl{?}                  & matches a single character\\
\cl{[...]}              & matches a character enclosed by the braces\\
\cl{\$\{\cv{var}\cl\}}  & value of the environment variable \cv{var}\\
\cl{\$(\cv{cmd})}       & output of \cv{cmd}\\
\cl{\$((\cv{expr}))}    & result of the mathematical expression \cv{expr}
\end{tabular}

\subsection*{Shell utilities}
\begin{tabular}{ll}
\shcmd{apropos}{text}       & searches the manual pages for \cv{text}\\
\shcmd{cat}{file}           & prints the contents of \cv{file}\\
\shcmd{cd}{dir}             & changes the \textit{working directory} to \cv{dir}\\
\shcmd{chmod}{prm file}     & changes permissions of \cv{file} to \cv{prm}\\
\shcmd{cp}{src dst}         & copies the file/directory \cv{src} to \cv{dst}\\
\shcmd{echo}{text}          & prints \cv{text}\\
\shcmd{file}{file}          & determines the file type of \cv{file}\\
\shcmd{find}{dir expr}      & finds files in \cv{dir} that match \cv{expr}\\
\shcmd{grep}{expr file}     & searches for pattern \cv{expr} in \cv{file}\\
\shcmd{ls}{dir}             & list the entries in the directory \cv{dir}\\
\shcmd{man}{cmd}            & displays the manual for \cv{cmd}\\
\shcmd{mkdir}{dir}          & creates the directory \cv{dir}\\
\shcmd{mv}{src dst}         & moves/renames \cv{src} to \cv{dst}\\
\shcmd{pwd}{}               & prints the current \textit{working directory}\\
\shcmd{rm}{file}            & removes the file \cv{file}\\
\shcmd{sort}{}              & sorts lines of text from input\\
\shcmd{touch}{file}         & creates the empty file \cv{file}
\end{tabular}
\subsection*{Input output redirection}
\begin{tabular}{ll}
\shcmd{\cv{cmd1}}{|} \shcmd{\cv{cmd2}}{}    & runs \cv{cmd1} and \cv{cmd2} and redirects the\\
                                            & output of \cv{cmd1} to the input of \cv{cmd2}\\
\shcmd{\cv{cmd}}{> file}                    & runs \cv{cmd} and redirects output to \cv{file},\\
                                            & content of \cv{file} is overwritten\\
\shcmd{\cv{cmd}}{>\null> file}              & like \shcmd{}{>>} but appends output to \cv{file}\\
\shcmd{\cv{cmd}}{< file}                    & runs \cv{cmd} and redirects \cv{file} to its input\\
\shcmd{\cv{cmd}}{<<< text}                  & runs \cv{cmd} with input \cv{text}
                                          
\end{tabular}
\subsection*{Job control}
\textit{Job} = Shell command and its associated process(es)
\begin{itemize}
    \item Each job has a job id and corresponding process ids
    \item Jobs can run in the foreground or in the background 
    \item The execution of a job can be temporarily suspended
\end{itemize}
\begin{tabular}{ll}
\shcmd{\cv{cmd}}{\&}    & starts \cv{cmd} as background job (id is printed)\\
\shcmd{fg}{\%job}       & puts the job \cv{job} in foreground\\
\shcmd{bg}{\%job}       & continues suspended job \cv{job} in background\\
\shcmd{jobs}{}          & prints job numbers of jobs in current terminal\\
\shcmd{kill}{pid}       & terminates a process with the process id \cv{pid}\\
\shcmd{ps}{}            & prints PIDs of processes in current terminal\\
\keys{Ctrl+S}/\keys{Q}  & suspends/resumes active job\\
\keys{Ctrl+Z}           & puts active job to background and suspends it\\
\keys{Ctrl+C}           & aborts the active job (most of the times)\\
\keys{Ctrl+D}           & sends an EOF character
\end{tabular}
\end{multicols*}
\end{document}
